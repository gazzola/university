%==============================================================================
\chapter{Trabalhos relacionados}\label{trabalhosrelacionados}
%==============================================================================

Vários métodos já foram propostos para resolver esse mesmo problema em português, apesar de nenhum deles ter um aproveitamento de 100\% de precisão, muitos deles já são utilizados em escala global em diversas ferramentas formuladas por grandes empresas de tecnologia [X,Y].

É apresentado em [X] um etiquetador morfo-sintático baseado em cadeias de Markov de tamanho variável. Ele é testado sobre o córpus Tycho Brahe [X], ele apresenta uma precisão de 95.51\%. 

Em [Y] é apresentado um etiquetador que aprende automaticamente as features a serem usadas através de uma rede neural profunda que emprega uma camada evolutiva capaz de aprender a uma representação de caracteres das palavras. Isso foi testado sobre diferente córpus, no Mac-Morpho [X] foi alcançado uma precisão de 97.47\%, no Tycho Brahe teve precisão de 97.17\%, além disso, foi feito testes sobre uma versão revisada do Mac-Morpho distribuída por [Y] e obteve uma precisão de 97.17\%

O mais recente etiquetador para o português brasileiro é mostrado em [X], onde é utilizado diferentes técnicas de representação das palavras e feita uma comparação entre elas. Nele é utilizado um modelo de rede neural profunda disponibilizado em [Y] e a melhor técnica obtém uma precisão de 97.47\% no Mac-Morpho original.