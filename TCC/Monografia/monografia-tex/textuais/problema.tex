%==============================================================================
\chapter{O problema}\label{oproblema}
%==============================================================================

Primeiramente, devemos informar que o problema a ser resolvido não é fácil, pois uma palavra pode possuir diferentes categorias gramaticais, e para solucioná-lo é  preciso, na maioria das vezes, analisar todo o contexto no qual a palavra está associada. Além disso, como já mostrado anteriormente, há muita ambiguidade no português do Brasil, e essa ambiguidade dificulta a análise morfo-sintática, porque não é possível determinar a priori qual classe gramátical a palavra pertence. No exemplo anterior, da \autoref{fig:exemploclassificacao}, não fica claro qual é o tipo de canto que está sendo referenciado, onde pode ser um verbo ou um substantivo (\textit{o ato de cantar na primeira pessoa do singular do presente; um subtantivo que se refere a um ângulo saliente; ou um substantivo que se refere a uma composição poetica}). Entretanto, a ambiguidade pode muitas vezes ser resolvida se o contexto for analisado, como por exemplo:

\begin{center}
\texttt{"O rio da cidade."}
\end{center}

A palavra \textit{rio} pode ser classificada como substantivo ou como verbo, mas levando em consideração o sentido da frase, fica evidente que a classificação correta é substantivo.

Uma estratégia trivial seria utilizar um longo dicionário com uma função de mapeamento de um para um, onde a \textit{chave} seria a palavra e o \textit{valor} seria a classe gramátical. Infelizmente essa técnica é inviável com os recursos computacionais que temos atualmente, visto que o número de entradas seria estupendamente grande por ter todas as palavras possíveis de português, caso contrário, haveria o
problema de ter uma palavra fora do vocabulário, e portanto ela não teria uma classe gramatical associada. Outro problema dessa estratégia é a ambiguidade, que faz com que uma palavra tenha mais que uma classe gramatical associada, e portanto não é possível mapear com certeza de que a classe associada é a correta sem antes de analisar o contexto.

Dito isso, o problema deste trabalho consiste em desenvolver um método para classificar palavras em suas respectivas classes gramaticais de modo eficiente. Para solucionar esse problema uma abordagem que está sendo amplamente utilizada é aprendizado de máquina, pois ela permite treinar um modelo que aprenda a classificar as palavras de acordo com o contexto em que está associada, ou seja, consegue realizar a análise morfo-sintática. 

Em ordem de conseguir solcuionar esse problema com eficiência, é necessário escolher um bom método computacional. Este trabalho se baseará na utilização de um método de aprendizado de máquina conhecido como redes neurais profundas (\textit{deep-learning}), para que seja possível treinar modelos capazes de realizar a classificação. Utilizamos redes neurais pois elas oferecem um jeito alternativo de realizar aprendizado de máquina quando temos hipóteses complexas com muitas características diferentes. 

