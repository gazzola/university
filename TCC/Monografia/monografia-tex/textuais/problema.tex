%==============================================================================
\chapter{O problema}\label{oproblema}
%==============================================================================

O problema a ser resolvido não é trivial, pois linguagens naturais tem bastante ambiguidade, onde uma palavra pode possuir diferentes categorias gramaticais. Além disso, há muita ambiguidade no português do Brasil, visto que é uma língua com uma sintaxe flexível e que possui uma rica morfologia. Essa ambiguidade dificulta a análise morfossintática, porque não é possível determinar a priori qual classe gramatical a palavra sendo analisada pertence. No exemplo anterior, da \autoref{fig:exemploclassificacao}, não fica claro qual é o tipo de \textit{banco} que está sendo referenciado, onde pode ser um verbo ou um substantivo (\textit{o ato de bancar na primeira pessoa do singular do presente; um subtantivo que se refere a um local onde armazena-se dinheiro; ou um substantivo que se refere um objeto para sentar.}). Para resolver o problema da ambiguidade, é necessário analisar os lexemas vizinhos de uma dada palavra, ou seja, é preciso analisar o seu contexto associado, por exemplo:


\begin{center}
\texttt{"O rio da cidade."}
\end{center}

A palavra \textit{rio} pode ser classificada como substantivo ou como verbo, mas levando em consideração o contexto, fica evidente que a classificação correta é substantivo.

Uma estratégia trivial seria utilizar um dicionário com uma função de mapeamento de um para um, onde a \textit{chave} seria a palavra e o \textit{valor} seria a classe gramatical. Infelizmente essa técnica requer muitos recursos computacionais, visto que o número de entradas seria estupendamente grande por ter todas as palavras possíveis de português, caso contrário, haveria o
problema de ter uma palavra fora do vocabulário, e portanto ela não teria uma classe gramatical associada. Porém, o principal revés dessa estratégia é a ambiguidade, que faz com que uma palavra tenha mais que uma classe gramatical associada, e portanto não é possível mapear com indubitabilidade de que a classe associada é a correta sem antes analisar o contexto.

Dito isso, este trabalho consiste em desenvolver um método para classificar palavras em suas respectivas classes gramaticais de modo eficiente. Uma abordagem que está sendo amplamente utilizada para resolver esse problema é aprendizado de máquina, pois ela permite treinar um modelo que aprenda a classificar as palavras de acordo com o contexto em que está associada. 

E em ordem de conseguir solucionar esse problema com eficiência, é necessário escolher um bom método computacional. Este trabalho se baseará na utilização de um método de aprendizagem profunda, que utiliza modelos neurais com múltiplas camadas (\textit{deep-learning}), para que seja possível treinar modelos capazes de realizar a classificação. Utilizamos redes neurais pois elas oferecem um jeito alternativo de realizar aprendizado de máquina quando temos hipóteses complexas com muitas características diferentes. 

