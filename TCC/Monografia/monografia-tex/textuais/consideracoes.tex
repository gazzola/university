%==============================================================================
\chapter{Considerações finais}\label{consideracoes}
%==============================================================================

A aprendizagem guiada tem como objetivo ganhar acurácia, pois com uma palavra já classificada, o contexto da janela de palavras sendo analisada fica mais representativo, e portanto menos ambíguo. Apesar de \citeonline{shen2007guided} não utilizar modelos baseados em redes neurais, conseguiu alcançar bons resultados. Além disso, estamos utilizando representações vetorizadas das classes gramaticais e das palavras, e com isso esperamos obter mais acurácia, pois estamos levando mais informações em consideração no momento em que a rede neural recorrente computa a hipótese. Ou seja, pretende-se obter bons resultados com essa metodologia.

A aplicação do método neural pode ser lento devido a utilização dos trigramas. Caso isso se confirme verdade, será feitos apenas alguns testes com eles e então diminuiremos a notação para funcionar com bigramas. 

Para esta etapa do trabalho, ainda não foram testadas as hipóteses levantadas através da metodologia, sendo assim, não há resultados previsórios.