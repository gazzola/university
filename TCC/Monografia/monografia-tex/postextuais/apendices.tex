\begin{apendicesenv}

% \lstdefinelanguage{Python}{
%  keywords={typeof, null, catch, switch, in, int, str, float, self, weights, activation, activation, optimizer, class_mode, mode, input_length},
%  keywordstyle=\color{ForestGreen}\bfseries,
%  ndkeywords={boolean, throw, import},
%  ndkeywords={return, class, if ,elif, endif, while, do, else, True, False , catch, def},
%  ndkeywordstyle=\color{BrickRed}\bfseries,
%  identifierstyle=\color{black},
%  sensitive=false,
% comment=[l]{\#},
%  morecomment=[s]{/*}{*/},
%  commentstyle=\color{purple}\ttfamily,
%  stringstyle=\color{red}\ttfamily,
% }

% Imprime uma página indicando o início dos apêndices
\partapendices

% Para cada apêndice, um \chapter


% %==============================================================================
% \chapter{Derivadas parciais do Gradiente Descendente}\label{app:derivadasgraddesc}
% %==============================================================================


\chapter{Implementação do modelo neural recursivo} \label{app:implementacaorecursivo}

\section{Arquitetura}

\inputminted[linenos,fontsize=\scriptsize]{python}{postextuais/recursive_model_network.py}

\section{Treinamento}

\inputminted[linenos,fontsize=\scriptsize]{python}{postextuais/recursive_model_train.py}

\section{Predição}

\inputminted[linenos,fontsize=\scriptsize]{python}{postextuais/recursive_model_predict.py}


\chapter{Implementação do modelo neural recorrente bidirecional} \label{app:implementacaorecorrentebidir}

\section{Arquitetura}

\inputminted[linenos,fontsize=\scriptsize]{python}{postextuais/recurrent_model_network.py}

\section{Treinamento}

\inputminted[linenos,fontsize=\scriptsize]{python}{postextuais/recurrent_model_train.py}

\section{Predição}

\inputminted[linenos,fontsize=\scriptsize]{python}{postextuais/recurrent_model_predict.py}


\end{apendicesenv}
