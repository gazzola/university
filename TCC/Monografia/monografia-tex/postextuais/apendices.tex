\begin{apendicesenv}

% Imprime uma página indicando o início dos apêndices
\partapendices

% Para cada apêndice, um \chapter


%==============================================================================
\chapter{Primeiro Apêndice}
%==============================================================================

De acordo com a ABNT:

\begin{quotation}
Apêndice (opcional): texto utilizado quando o autor pretende complementar sua argumentação. São identificados por letras maiúsculas e travessão, seguido do título. Ex.: APÊNDICE A - Avaliação de células totais aos quatro dias de evolução

Anexo (opcional): texto ou documento \textbf{não elaborado pelo autor} para comprovar ou ilustrar. São identificados por letras maiúsculas e travessão, seguido do título. Ex.: ANEXO A - Representação gráfica de contagem de células
\end{quotation}

Tais definições (e outras) podem ser encontradas na NBR 14724-2001 Informação e documentação - trabalhos acadêmicos\footnote{http://www.firb.br/abntmonograf.htm}.


%==============================================================================
\chapter{Segundo Apêndice}
%==============================================================================

Pode ser que tenha outro...


\end{apendicesenv}
