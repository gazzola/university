\begin{resumo}[Abstract]
 
Part-of-speech Tagging consists in classify a given word, that belongs to a collections of texts, with particular part of speech tag, based in its context.  Part-of-speech Tagging can be used as pre-processing of many applications, so, in Natural Language Processing we are always searching for improvement methods. We study different methods to generate words representations (word embeddings) from a collection of texts denominated corpora. We create a deep learning model by using a recurrent neural network that receives as input the word embeddings and predicts a part-of-speech tag to the associated word. The neural model is guided, where easy words to classify are applied first in the network. The output is a vector tag, that is then complemented with the input of the same word classified. The training process occur over three different tagged corpora for brazilian portuguese.

 \vspace{\onelineskip}
 
 \noindent 
 \textbf{Key-words}: Machine learning. Deep learning. Natural language processing. Part-of-speech tagging. Neural networks. Word embeddings.
\end{resumo}
