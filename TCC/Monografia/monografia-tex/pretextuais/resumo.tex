\begin{resumo}
 
Part-of-speech Tagging consiste em classificar uma palavra pertencente a um conjunto de textos em uma classe gramatical através da análise morfossintática. Como Part-of-speech Tagging pode ser aplicada como pré-processamento de várias aplicações, em Processamento de Linguagem Natural estamos sempre buscando métodos que realizam esse processo com mais desempenho. Nós estudamos diferentes métodos para gerar representações de palavras \textit{word embeddings} a partir de uma coleção de textos denominadas \textit{córpus}. Criamos então um modelo baseado em aprendizagem profunda utilizando uma rede neural recorrente que recebe essas representações como entrada e gerá como saída uma classe gramatical associada. O modelo neural é guiado, onde palavras mais fáceis de serem analisadas são aplicadas primeiro na rede. A saída é uma classe vetorizada, que é complementada na entrada da própria palavra classificada. O treinamento é feito sobre três diferente \textit{córpus} etiquetados para o português brasileiro.

 \vspace{\onelineskip}
    
 \noindent
 \textbf{Palavras-chave}: Aprendizado de máquina. Aprendizagem profunda. Processamento de linguagem natural. Part-of-speech tagging. Redes neurais. Word embeddings.
\end{resumo}
