\begin{resumo}
 
Part-of-speech Tagging consiste em classificar uma palavra pertencente a um conjunto de textos em uma classe gramatical. Em Processamento de Linguagem Natural estamos sempre buscados métodos modernos para o processo de Part-of-speech Tagging, pois ele pode ser usado como pré-processamento de várias aplicações. Estudamos diferentes métodos para gerar representações de palavras (\textit{word embeddings}) a partir de uma coleção de textos denominada \textit{córpus}. Propomos então um modelo baseado em aprendizagem profunda utilizando uma rede neural recursiva que recebe essas representações como entrada e gera como saída uma classe gramatical associada. O modelo neural é guiado, onde palavras mais fáceis de serem analisadas são classificadas primeiro. A saída da rede é uma classe vetorizada, que é complementada na entrada com o vetor da própria palavra classificada. O treinamento é feito sobre três diferente \textit{córpus} etiquetados para o português brasileiro.

 \vspace{\onelineskip}
    
 \noindent
 \textbf{Palavras-chave}: Aprendizado de Máquina. Aprendizagem Profunda. Processamento de Linguagem Natural. Part-of-speech Tagging. Redes Neurais Recursivas. Word Embeddings.
\end{resumo}


