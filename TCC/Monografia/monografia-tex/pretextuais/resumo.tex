\begin{resumo}
 
Part-of-speech Tagging consiste em classificar uma palavra pertencente a um conjunto de textos em uma classe gramatical. Em Processamento de Linguagem Natural estamos sempre buscando métodos modernos para o processo de Part-of-speech Tagging, pois ele pode ser usado como pré-processamento de várias aplicações. Estudamos diferentes métodos para gerar representações de palavras (\textit{word embeddings}). Propomos dois modelos baseado em aprendizagem profunda: um modelo neural recursivo e um modelo neural recorrente bidirecional. O modelo neural recursivo é guiado por palavras mais fáceis de serem classificadas. O treinamento foi feito sobre três diferentes \textit{córpus} etiquetados para o português brasileiro. Para cada modelo, avaliamos a acurácia sobre esses \textit{córpus} utilizando três tipos de representações de palavras, e fizemos uma análise dos erros cometidos por eles. Além disso, comparamos os resultados de acurácia com os trabalhos relacionados e constatamos que nosso modelo recorrente bidirecional conseguiu a segunda melhor acurácia sobre o Mac-Morpho original para palavras fora do vocabulário. Nossos experimentos mostram que o modelo neural recorrente bidirecional é mais eficiente que o recursivo em termos de acurácia e tempo de treinamento. Este trabalho contribui com a definição e implementação de dois etiquetadores que foram disponibilizados à comunidade e que podem ser usados livremente.

 \vspace{\onelineskip}
    
 \noindent
 \textbf{Palavras-chave}: Aprendizagem Profunda. Processamento de Linguagem Natural. Part-of-speech Tagging. Redes Neurais Recursivas. Redes neurais recorrentes bidirecionais.
\end{resumo}


