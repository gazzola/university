\documentclass[10pt]{beamer}


\usetheme[nosectionslide,everytitleformat=regular]{m}

% camel case
% \renewcommand{\mthemetitleformat}{}

\definecolor{mDarkRed}{HTML}{8F0129}
\definecolor{mGreen}{HTML}{0E8740}

\definecolor{BrickRed}{HTML}{B6321C}
\definecolor{DarkOrchid}{HTML}{A4538A}
\definecolor{ForestGreen}{HTML}{009B55}
\definecolor{Goldenrod}{HTML}{897E4D}

\setbeamercolor{alerted text}{fg=mGreen}
\setbeamertemplate{bibliography item}[text]

\setbeamertemplate{footnote}{\hangpara{2em}{1}\makebox[2em][l]{\insertfootnotemark}\footnotesize\insertfootnotetext\par}


\usepackage[scale=2]{ccicons}

\usepackage{cmap}           % Mapeamento de caracteres especiais no PDF
\usepackage{lmodern}        % Usa fonte Latin Modern
\usepackage[T1]{fontenc}    % Seleção de codificação de fonte
\usepackage[utf8]{inputenc} % Codificação do arquivo (conversão automática dos acentos)
\usepackage[brazil]{babel}  % Idioma para hifenização e tradução de vários elementos
\usepackage{makeidx}        % Criação de índice
\usepackage{hyperref}       % Formatação do índice
\usepackage{lastpage}       % Usado pela Ficha catalográfica
\usepackage{indentfirst}    % Indenta o primeiro parágrafo de cada seção
\usepackage{graphicx}       % Inclusão de gráficos
\usepackage{booktabs}       % Formatação de tabelas
% -------------------------------------------------------------------------------------------------
% Para citações
% \usepackage[brazilian,hyperpageref]{backref} % Páginas com as citações na bibliografia
\usepackage[alf]{abntex2cite} % Citações padrão ABNT (alfanumérico)
% -------------------------------------------------------------------------------------------------
% Pacotes opcionais
\usepackage{nomencl}        % Para criar uma lista de símbolos
\usepackage{acro}           % Para usar acrônimos e abreviaturas
\usepackage{tikz}           % Para fazer figuras, diagramas e gráficos integrados e elegantes
\usepackage{pgfplots}       % Usa o pacote tikz para fazer gráficos muito melhores que os do Excel
\usepackage{pgfplotstable}  % Para gerar tabelas automaticamente a partir de arquivos com dados
\usepackage{filecontents}   % Para colocar o conteúdo de um arquivo dentro de um arquivo tex
\usepackage{todonotes}      % Para criar anotações durante o desenvolvimento do texto
%\usepackage{multirow}       % Permite fazer tabelas com múltiplas linhas
\let\newfloat=\undefined    % Workaround para usar o pacote algorithm
\usepackage{algorithm}      % Para escrever algoritmos
\usepackage{algpseudocode}
\usepackage{pifont}
% \usepackage{clrscode}       % Para escrever algoritmos
% \usepackage{clrscode3e}     % Para escrever algoritmos; mais simples que os pacotes acima
\usepackage{pdfpages}        % Para incluir a folha de aprovação assinada em PDF
\usepackage{amsmath}
\usepackage{amsfonts}
\usepackage{mathtools}
\usepackage{subcaption}
\newcolumntype{P}[1]{>{\centering\arraybackslash}m{#1}} 

\usepackage{tablefootnote}
\usepackage{adjustbox}
\usepackage{changepage}
\usepackage{color}

% \usepgfplotslibrary{external} 
% \tikzexternalize


\captionsetup{labelformat=empty,labelsep=none}
\def\multiset#1#2{\ensuremath{\left(\kern-.3em\left(\genfrac{}{}{0pt}{}{#1}{#2}\right)\kern-.3em\right)}}

\makeatletter
\newcommand\footnoteref[1]{\protected@xdef\@thefnmark{\ref{#1}}\@footnotemark}
\makeatother

\DeclareMathOperator*{\argmin}{arg\,min}
\DeclareMathOperator*{\argmax}{arg\,max}


\title{Classificação de SPAM usando Naive Bayes e SVM}
\subtitle{}
\date{07 de novembro de 2015}
\author[Treviso]{Marcos Treviso}
\institute{Aprendizado de Máquina - Universidade Federal do Pampa}
\titlegraphic{\hfill\includegraphics[height=1.25cm]{img/unipampa_logo.png}}

\begin{document}

\maketitle


\begin{frame}
  \frametitle{Roteiro}

  \begin{itemize}

    \item Multinomial Naive Bayes
    
    \item Ambiente de Teste

    \item Resultados

    \item Conclusão

  \end{itemize}

\end{frame}



\begin{frame}
  \frametitle{Roteiro}

  \begin{itemize}

    \item Multinomial Naive Bayes

    \item[\color{gray}{$\bullet$}] Ambiente de Teste

    \item[\color{gray}{$\bullet$}] \textcolor{gray}{Resultados}

    \item[\color{gray}{$\bullet$}] \textcolor{gray}{Conclusão}

  \end{itemize}

\end{frame}




\section{Multinomial Naive Bayes}

\begin{frame}[fragile]
  \frametitle{Multinomial Naive Bayes}

 \begin{itemize}

      \item Indicado para classificação de documentos

      \item[\ ] \ 

      \item Captura a frequência das palavras

      \item[\ ] \ 

      \item Assume indepência entre palavras
  \end{itemize}

  
\end{frame}

\begin{frame}[fragile]
  \frametitle{Multinomial Naive Bayes}

  \begin{equation} \nonumber
    P(x|y) = \prod_{j=i}^{V} P(w_j|y)^{n_j(x)} 
    \end{equation}

    \begin{equation} \nonumber
    \hat{P}(y_k) = \frac{|J_k|}{M} \mbox{, } \qquad \hat{P}(w_j | y_k) = \frac{\sum_{m \in J_k} n_j(x^m)}{\sum_{i=1}^{V}\sum_{m \in J_k} n_i(x^m)}
    \end{equation}

 \begin{itemize}
      \item Aplicando $log$: soma de probabilidades
      \item $M$ = número de documentos
      \item $V$ = tamanho do vocabulário
      \item $x^i$ = i-ésimo documento
      \item $y^i$ = classe de $x^i$
      \item $J_k$ = indíces das instâncias pertencendo a k-ésima classe
      \item $n_j(x)$ = número de occorências da palavra $w_j$ no documento $x$
   \end{itemize}

\end{frame}


\begin{frame}
  \frametitle{Roteiro}

  \begin{itemize}

    \item[\color{gray}{$\bullet$}] \textcolor{gray}{Multinomial Naive Bayes}

    \item Ambiente de Teste

    \item[\color{gray}{$\bullet$}] \textcolor{gray}{Resultados}

    \item[\color{gray}{$\bullet$}] \textcolor{gray}{Conclusão}

  \end{itemize}

\end{frame}



\section{Ambiente de Teste}

\begin{frame}[fragile]
  \frametitle{Ambiente de Teste}

 \begin{itemize}
      \item Dell Inspiron 7348
    \end{itemize}

    \begin{table}[!htb]
    \footnotesize
    \centering
    \begin{tabular}{ll}
      \toprule
      \textbf{Processador} & Intel i7-5500U 2.40GHz - 4 cores  \\
      \textbf{Memória Principal} & 8 GB RAM DDR3 \\
      \textbf{Cache} & L3 Cache 4 MB \\
      \midrule
      \textbf{Processador} & Intel Xeon X5690 3.40GHz - 24 cores  \\
      \textbf{Memória Principal} & 64 GB RAM DDR3 \\
      \textbf{Cache} & L3 Cache 12 MB \\
      \bottomrule
    \end{tabular}
    \end{table}

  \begin{itemize}
    \item Python 3.4.3
    \item Numpy 1.9.2
  \end{itemize}
  
\end{frame}


\begin{frame}[fragile]
  \frametitle{Ambiente de Teste}

  \begin{itemize}
    \item Multinomial Naive Bayes:
    \begin{itemize}
      \item[-] $smooth$ de $10^{-9}$
      \item[-] Palavras em $lowercase$
      \item[-] Números para $NUMBER$
      \item[-] Endereços web para $URL$
      \item[-] Emails web para $EMAIL$
      \item[-] \$ web para $DOLLAR$
      \item[-] Remoção de HTML e pontuação
    \end{itemize}

    \item[\ ] \ 
  \end{itemize}


  \begin{itemize}
    \item Redes Neurais
    \begin{itemize}
      \item[-] Regularização de $0.1$
      \item[-] Palavras em $lowercase$
      \item[-] Números para $NUMBER$
      \item[-] Endereços web para $URL$
      \item[-] Emails web para $EMAIL$
      \item[-] \$ web para $DOLLAR$
      \item[-] Remoção de HTML e pontuação
    \end{itemize}
  \end{itemize}



\end{frame}


\begin{frame}
  \frametitle{Roteiro}

  \begin{itemize}

    \item[\color{gray}{$\bullet$}] \textcolor{gray}{Multinomial Naive Bayes}

    \item[\color{gray}{$\bullet$}] \textcolor{gray}{Ambiente de Teste}

    \item Resultados

    \item[\color{gray}{$\bullet$}] \textcolor{gray}{Conclusão}

  \end{itemize}

\end{frame}



\section{Resultados}


\begin{frame}
 \frametitle{Resultados - Acurácia}

 \begin{itemize}

    \item Média de 10 execuções
  \end{itemize}


  \begin{table}[!htb]
      \footnotesize
      \centering
      \begin{tabular}{l|cc|cc|cc}
      \toprule
      Alg. & \multicolumn{2}{c}{Enron1} & \multicolumn{2}{|c}{Enron2} & \multicolumn{2}{|c}{Enron3} \\
      \toprule
      \     & Treino & Teste     & Treino & Teste    &  Treino & Teste    \\
      \midrule
      MNV &    99.25\% &  98.56\%   & 99.04\% & 98.64\%   & 98.88\% & 98.53\%  \\
      SVM &    98.68\% &  98.12\%   & 98.28\% & 97.87\%   & 98.27\% & 97.94\%   \\
      \bottomrule
      \end{tabular}

      \end{table}


    \begin{table}[!htb]
      \footnotesize
      \centering
      \begin{tabular}{l|cc|cc}
      \toprule
      Alg. & \multicolumn{2}{c}{Enron4} & \multicolumn{2}{|c}{Enron5} \\
      \toprule
      \     & Treino & Teste     & Treino & Teste   \\
      \midrule
      MNV &    99.12\% &  98.83\%   & 99.33\% & 99.18\% \\
      SVM &    98.26\% &  96.10\%   & 97.44\% & 96.17\%  \\
      \bottomrule
      \end{tabular}

      \end{table}


\end{frame}



\begin{frame}
 \frametitle{Resultados - Pontuação F1}

 \begin{itemize}
    \item Média de 10 execuções
  \end{itemize}


  \begin{table}[!htb]
      \footnotesize
      \centering
      \begin{tabular}{l|cc|cc|cc}
      \toprule
      Alg. & \multicolumn{2}{c}{Enron1} & \multicolumn{2}{|c}{Enron2} & \multicolumn{2}{|c}{Enron3} \\
      \toprule
      \     & Treino & Teste     & Treino & Teste    &  Treino & Teste    \\
      \midrule
      MNV &    0.99\% &  0.99\%   & 0.98\% & 0.98\%   & 0.98\% & 0.99 \%  \\
      SVM &    0.96\% &  0.95\%   & 0.94\% & 0.91\%   & 0.94\% & 0.89\%   \\
      \bottomrule
      \end{tabular}

      \end{table}


    \begin{table}[!htb]
      \footnotesize
      \centering
      \begin{tabular}{l|cc|cc}
      \toprule
      Alg. & \multicolumn{2}{c}{Enron4} & \multicolumn{2}{|c}{Enron5} \\
      \toprule
      \     & Treino & Teste     & Treino & Teste   \\
      \midrule
      MNV &    0.99\% &  0.97\%   & 0.98\% & 0.97\% \\
      SVM &    0.97\% &  0.95\%   & 0.97\% & 0.96\%  \\
      \bottomrule
      \end{tabular}

      \end{table}


\end{frame}



\begin{frame}
  \frametitle{Roteiro}

  \begin{itemize}

    \item[\color{gray}{$\bullet$}] \textcolor{gray}{Multinomial Naive Bayes}

    \item[\color{gray}{$\bullet$}] \textcolor{gray}{Ambiente de Teste}

    \item[\color{gray}{$\bullet$}] \textcolor{gray}{Resultados}

    \item Conclusão

  \end{itemize}

\end{frame}


\section{Conclusão}

\begin{frame}[fragile]
  \frametitle{Conclusão}

  \begin{itemize}

    \item MNB parece ser muito bom para a tarefa

    \item MNB é bem mais rápido para treinar que o SVM

    \item A acurácia de ambos ficou boa

    \item Os filtros de pré-processamento ajudam a deixar o probleam mais fácil

    \item O código do trabalho está em: \url{https://github.com/meitcher/spam-classification}

  \end{itemize}

\end{frame}




\begin{frame}[fragile]
  \frametitle{Referências}

  \begin{itemize}

    \item LxMLS - Lab Guide, Jul, 19, 2015. Lisboa, Portugal.

  \item Machine Learning, Andrew Ng. Coursera.
  \end{itemize}
\end{frame}




\plain{Classificação de SPAM usando Naive Bayes e SVM \\ \ \\ \normalsize{Marcos Treviso} \\ \ \\ \ \\ {Aprendizado de Máquina} \\ \ \\ \ \\ {\normalsize 7 de novembro de 2015} \\ {\scriptsize Universidade Federal do Pampa}}

\end{document}
